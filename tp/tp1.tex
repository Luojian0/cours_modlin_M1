% -*- TeX -*- -*- FR -*-
\documentclass[11pt,a4paper]{article}

%\usepackage{isolatin1}
\usepackage[T1]{fontenc}
\usepackage[utf8]{inputenc}
\usepackage{latexsym}
\usepackage{amsmath}
\usepackage{amssymb}
\usepackage{amsthm}
\usepackage{graphicx,epsfig}
\usepackage[french]{babel}
\usepackage{fancyvrb}
\usepackage{xcolor}
\usepackage{enumitem}

%%%%%%%%%% Marges
\oddsidemargin 0cm \evensidemargin -1cm \topmargin -1cm
\textheight 24cm \textwidth 15cm


%%%%%%%%%% Raccourcis pour les outils
\newcommand{\argmin}{\displaystyle \mathop{argmin}}
\newcommand{\tend}{\displaystyle \mathop{\longrightarrow}}
\newcommand{\somme}{\displaystyle \sum}
\newcommand{\mytilde}{$\sim$}
\newcommand{\fleche}{\texttt{<-}}%$\mathbf{<\!\!-}$\;}

%%%%%%%%%% C'est plus beau comme �a...
\renewcommand{\leq}{\leqslant}
\renewcommand{\geq}{\geqslant}
\renewcommand{\epsilon}{\varepsilon} 
\renewcommand{\phi}{\varphi}

%%%%%%%%% Raccourcis usuels
\def\1{\mbox{1\hspace{-.35em}1}}
\def\R{\mathbb{R}}
\def\B{\mathbb{B}}
\def\N{\mathbb{N}}
\def\P{\mathbb{P}}
\def\E{\mathbb{E}}
\def\L{\mathbb{L}}
\def\F{\mathbb{F}}
\def\R{\mathbb{R}}
\def\C{\mathbb{C}}
\def\Z{\mathbb{Z}}
\def\G{\mathbb{G}}
\def\D{\mathbb{D}}
\def\H{\mathbb{H}}
\def\v{\mbox{Var\,}}
\def\lip{\mbox{Lip\,}}
\def\Lip{\mbox{Lip\,}}
\def\cov{\mbox{Cov}}%

\newtheorem{Theoreme}{Th\'eor\`eme}%[section]
\newtheorem{Proposition}{Proposition}[]%[Theoreme]{Proposition}
\newtheorem{Corollaire}[Proposition]{Corollaire}
\newtheorem{Lemme}[Proposition]{Lemme}
%\newtheorem{Definition}[Theoreme]{D\'efinition}
\newcommand{\findem}{\hfill\hbox{\hskip 4pt
\vrule width 2pt height 6pt depth 1.5pt}\vspace{.5cm}\par}

\newenvironment{code}{ \VerbatimEnvironment \begin{Verbatim}[commandchars=\\\{\}]}{\end{Verbatim}}

\setlength{\parindent}{0pt}
\setlength{\parskip}{0pt}

   \newtheorem{exercice}{\textbf{Exercice}}


%%%%%%%%%%


%\newcommand{\ms}{\noalign{\vspace{3\p@ plus2\p@ minus1\p@}}}
%\newcommand{\boldarrayrulewidth}{0.5\p@}
%\newcommand{\midarrayrulewidth}{0.25\p@}


% Weighted rules for tables
%\newcommand{\toprule}{\ms\topline\ms}
%\newcommand{\midrule}{\ms\midline\ms}
%\newcommand{\bottomrule}{\ms\topline}


%%%%%%%%%%%%%%%%%%%%%%%
%\pagestyle{fancy}
%\lhead{ }
%\rhead{ }
%\} {0cm}{1cm}{0cm}{1cm}


\begin{document}

\noindent Modèle Linéaire
\hspace{10.5cm}
2016-2017




\begin{center}
\textbf{TP -- Régression linéaire simple}

\end{center}

\vspace{1cm}


\begin{exercice}
\end{exercice}
Récupérer la base \verb+anscombe+.
\begin{code}
> data(anscombe)
> attach(anscombe)
\end{code}
\begin{enumerate}
\item Réaliser l'ajustement d'un modèle linéaire simple pour chacun des 4 jeux de données \verb+(x1,y1)+, \verb+(x2,y2)+, \verb+(x3,y3)+ et \verb+(x4,y4)+ à l'aide de la fonction \verb+lm+. On ne demande pas de réaliser l'étude des résidus.
\item Comparer les principales statistiques de l'ajustement, données par la fonction \verb+summary+.
\item Faire les représentations graphiques de \verb+(x1,y1)+, \verb+(x2,y2)+, \verb+(x3,y3)+ et \verb+(x4,y4)+ ainsi que des droites de régression linéaire obtenues. Commenter.
\end{enumerate}

\vspace{1cm}


\begin{exercice}
\end{exercice}
On utilise la base \verb+cats+ de la library \verb+MASS+.
\begin{code}
> library(MASS)
> attach(cats)
\end{code}
On souhaite expliquer le poids du coeur (Hwt) par le poids du corps (Bwt)

\begin{enumerate}
	\item Tracer un nuage de points représentant le poids du coeur en fonction du poids du corps.
	\item On cherche à étudier la relation linéaire entre ces deux variables. On utilise pour cela la fonction \verb+lm+ de R. 
\begin{code}
> cats.lm \fleche lm(Hwt \mytilde Bwt)
> summary(cats.lm)
\end{code}
Commenter les résultats obtenus.
\item Tracer la droite de régression estimée par le modèle linéaire sur le nuage de points.
\item Faire l'étude des résidus.

\end{enumerate}


\vspace{1cm}


\begin{exercice}
\end{exercice}
On utilise la base \verb+cars+ de \verb+R+. Ce paquet contient deux variables : la vitesse et la distance de freinage de 50 voitures, mesurées en 1920. On souhaite expliquer la distance de freinage \verb+dist+ par la vitesse de la voiture \verb+speed+.

\begin{enumerate}
	\item Tracer un nuage de points représentant la distance de freinage en fonction de la vitesse de la voiture.
	\item On cherche à étudier la relation linéaire entre ces deux variables à l'aide de la fonction \verb+lm+ de R. Commenter les résultats obtenus.
	\item Tracer la droite de régression estimée par le modèle linéaire sur le nuage de points.
\item Faire les représentations graphiques des résidus studentisés et tester les hypothèses du modèle.
\item Représenter les intervalles de confiance et de prédiction.
\item On souhaite manitenant expliquer la distance de freinage par le carré de la vitesse. On introduit \verb+speed2 <- speed^2+. Faire la représentation graphique et ajuster le modèle avec une contrainte de contante nulle : $dist_i=a\cdot speed2_i+\epsilon_i$. Pour cela on utilisera la commande \texttt{cars.lm2 \fleche~lm(dist\mytilde 0+speed2)}. Comparer les résultats obtenus avec le premier modèle.

\end{enumerate}




\vspace{1cm}


\begin{exercice}
\end{exercice}
On utilise la base \verb+p9.10+ du paquet \verb+MPV+. L'objectif est d'expliquer l'usure de l'asphalte \verb+y+ en fonction des caractéristiques du revêtement. Nous nous intéresserons plus spécifiquement à l'influence de la viscosité, donnée par \verb+x1+ en échelle logarithmique.

\begin{enumerate}
	\item Faire l'ajustement du modèle linéaire de \verb+y+ par rapport à \verb+x+. Vérifier les hypothèses du modèle. Qu'en concluez-vous ?
	\item Au vu des représentations graphiques (\verb+pairs(p9.10)+) on souhaite faire intervenir dans le modèle la variable \verb+x4+. Cette variable est un indicateur prenant les valeurs \verb+-1+ et \verb|+1|.
	\begin{enumerate}
	\item Ajuster un modèle de régression linéaire sur le sous-ensemble \verb+x4==-1+. On pourra utiliser l'option \verb+subset+ de \verb+lm+. Faire l'étude des résidus.
	\item Ajuster un modèle de régression linéaire sur le sous-ensemble \verb|x4==+1|. Faire l'étude des résidus.
\end{enumerate}	 
	

\end{enumerate}

%\begin{exercice}
%\end{exercice}
%On utilise la base \verb+khct+ du paquet \verb+caschrono+. On récupère deux variables : la consommation électrique mensuelle en kilo-Watt-heures et \Verb+temperature+, qui correspond à la somme sur le mois des $(temp_i-65)_+$ où $temp_i$ est la température moyenne sur une journée, en degré Farenheit (65$^o$F correspond environ à 18$^o$C).
%\begin{code}
%> library(caschrono)
%> data(khct)
%> conso <- khct[,"kwt"]
%> temperature <- khct[,"cldd"]
%\end{code}
%On souhaite expliquer  la consommation électrique en kWh \verb+conso+ en fonction de la variable \verb+temperature+ à l'aide d'un modèle linéaire simple gaussien.
%
%Faire l'analyse.
%
%\vspace{1cm}
%
%\begin{exercice}
%\end{exercice}
%On utilise la base \verb+airquality+ de R.
%\begin{code}
%> attach(airquality)
%\end{code}
%On souhaite expliquer le taux d'ozone \verb+Ozone+ par la vitesse du vent \verb+Wind+.
%
%Faire l'analyse.


%
%\vspace{1cm}
%
%\begin{exercice}
%\end{exercice}
%On utilise la base \verb+co2+ de \verb+R+. On récupère deux variables : les dates de mesures et la consommation électrique à ces dates.
%\begin{code}
%> library(caschrono)
%> data(khct)
%> conso <- khct[,"kwt"]
%> temps <- time(khct)
%\end{code}
%On souhaite expliquer  la consommation électrique en kWh \verb+conso+ en fonction du temps \verb+temps+ à l'aide d'un modèle linéaire simple gaussien.
%
%Faire l'analyse.





%\vspace{1em}
%\textbf{Options de la fonction} \verb+plot+ 
%\vspace{0.5em}


%\vspace{-15pt}
%\begin{table}[h]
%\begin{tabular} {l l}
%\verb+main="Le titre"+ & titre du graphe \\
%\verb+sub="Le sous-titre"+ & sous-titre du graphe \\
%\verb+xlab="",ylab=""+ & légende des axes \\
%\verb+xlim=c(xa,xb)+ & limites des axes \\
%\verb+ylim=c(ya,yb)+ & \\
%\verb+type=""+ & par défaut : "p" pour points\\
%&sinon "l", "o", "h" ou "s" \\
%\verb+log=""+ & échelle logarithmique pour les X ("x"),\\
%&  pour les Y ("y") ou les 2("xy")\\
%\verb+pch=n, lty=n+ & symbole ou type de ligne utilisé \\
%\verb+col=n+ & couleur\\
%\verb+font=n,cex=n+ & fonte et taille du texte et des symboles \\
%%{\sf } & \\

%\end{tabular}
%\end{table}
%%\vspace{-5pt}
%Pour plus de détails, voir \verb+help(par)+

\end{document}

% -*- TeX -*- -*- FR -*-

